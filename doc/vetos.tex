\documentclass[10pt,a4paper]{article}
\usepackage[utf8]{inputenc}
\usepackage{polski}
\usepackage{amsmath}
\usepackage{amsfonts}
\usepackage{amssymb}
\usepackage{graphicx}
\usepackage{hyperref}
\usepackage{listings}


\author{Klaudia Czuba, Mateusz Koprucki}
\title{VetOS - system wsparcia kliniki weterynaryjnej}
\date{}
\begin{document}

	
	\maketitle
	\newpage
	\tableofcontents
	\newpage
	
	\section{Wstęp}
	Przedmiotem dokumentacji jest system wsparcia kliniki weterynaryjnej przygotowany na zajęcia laboratoryjne z systemów baz danych. Autorami systemu są:
		\begin{itemize}
			\item Klaudia Czuba
			\item Mateusz Koprucki
		\end{itemize}
	System został zaprojektowany w formie aplikacji webowej do stworzenia której użyto następujących technologii:
		\begin{itemize}
			\item Serwer www - Apache2 wraz z PHP
			\item Serwer bazodanowy - Mysql 5.6.15
			\item System zarządzania treścią (CMS) Wordpress 4.7 z rozszerzeniami
		\end{itemize}
\newline
	Do celów demonstracyjnych aplikacja posiada trzy konta:
		\begin{itemize}
			\item vet\_test, hasło vet\_test - do prezentacji grupy weterynarzy
			\item staff\_test, hasło staff\_test - do prezentacji grupy obsługi
			\item admin\_test, hasło  admin\_test - do prezentacji możliwości administracji
		\end{itemize}
	W dokumentacji zamieszczono kody źródłowe SQL ze zmiennymi PHP np. 
	\begin{itemize}
		\item '.\$zmienna.'
		\item '\$zmienna'
	\end{itemize}
\newline		
		Pełne repozytorium dostępne jest pod adresem: 
	
	\url{https://github.com/zaknaifen/vetos_core}
	
	\section {Instalacja}
		W celu zainstalowania systemu należy zainstalować na serwerze lub komputerze lokalnym środowisko
		\begin{itemize}
			\item Dla Windows: Webserv 2.2
			\item Dla Linux: apache2 z php i mariadb.
		\end{itemize}
	Następnie ściągnąc repozytorium z adresu podanego we wstępie do lokalizacji środowiska. Z folderu sql wgrać poprzez phpmyadmin lub konsole mysql plik sql z bazą danych. 
	
	
	
	
	
	\section {Użytkownicy}
		Aplikacja pozwala na przypisanie do systemu wiele kont przez administrację z podziałem na grupy zdefinowane przez twórców aplikacji. Są to:
		\begin{itemize}
			\item Weterynarz (vet) - grupa przeznaczona dla lekarzy, pozwala na dostęp do modułu weterynaryjnej aplikacji
			\item Obsługa (staff) - grupa przeznaczona dla obsługi z dostępem do sekcji modułu obsługi
			\item Administratcja (admin) - grupa administracyjna, pełen dostęp do modułów. 
		\end{itemize}
	
	\section{Moduły}
	Aplikacja została podzielona na trzy główne moduły. 
	\subsection{Moduł weterynarza}
	Moduł weterynarza jest dostępny dla użytkowników z grupy weterynarzy i administracji. Sekcja została podzielona na dwie części:
	\begin{itemize}
		\item $[W$] Grafik - przyszłe wizyty weterynarza, wraz z formularzem przejścia do konkretnej wizyty. 
		
		OBRAZEK!
		
		Po wprowadzeniu numeru wizyty aplikacja przenosi użytkownika na stronę realizacji wizyty, gdzie:
			\begin{itemize}
				\item W częsci górnej pojawiają się dane pacjenta wraz z powodem wizyty
				\item W centralnej części znajduję się arkusz do wypełnienia podczas wizyty
					\begin{itemize}
						\item Diagnoza
						\item Zalecenia
					\end{itemize}
				\item Z prawej strony wyświetlają się 4 ostatnie wpisy z historii wizyt pacjenta z możliwością przejścia do pełnej historii.
			
			\end{itemize}
		\end{itemize}
		OBRAZEK!
		
		SQL Wykorzystany dla tej strony:
		
		\begin{lstlisting}
		select 
		a.vet_id,
		a.event_id as `numer wizyty`,
		a.event_begin as `data`,
		a.event_time as `godzina`,
		a.event_desc as `opis`,
		b.patient_id as `numer pacjenta`,
		b.patient_name as `imie pacjenta`,
		c.owner_name as `imie  wlasciciela`,
		c.owner_surname as `nazwisko wlasciciela`
		
		from wp_calendar a
		join patient_info b on a.patient_id=b.patient_id
		join owner_info c on b.patient_id=c.owner_id
		where 
		(event_time>=curtime() 
		and a.vet_id=(select ID from wp_users 
		  where user_login="'.$login.'") 
		and event_begin>=curdate())
		
		or 
		(event_begin>=sysdate()
		and a.vet_id=(select ID from wp_users
		  where user_login="'.$login.'"))
		
		order by event_begin asc;
		\end{lstlisting}
		
		
		
\end{document}